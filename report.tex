\documentclass[a4paper,10pt]{article}
\usepackage[hscale=0.7,vscale=0.5]{geometry}
\usepackage{lipsum}
\usepackage{fullpage}

\pagenumbering{gobble}

% Title Page

\twocolumn

\begin{document}
\section{Introduction}
This document describes our efforts, steps and results in creating
a content-based video search engine. That is to search a database
of videos using an image or audio fragment as 'search term' and to
search the acual video content(frames or audio of the video).

The following tasks have been accomplished in order to create
our search engine:
\begin{enumerate}
  \setlength{\itemsep}{1pt}
  \setlength{\parskip}{1pt}
  \setlength{\parsep}{0pt}
 \item Extract frames and audio from a large amount of videos in a distributed way using Hadoop.
 \item From those extracted frames and audio, extract low level features.
 \item Create a searcher which can search related images/audio fragments by using the low level features.
 \item Create a web interface accepts as input an image or an audio fragment and displays related videos.
\end{enumerate}


\section{Goal}
Through this experiment we wish the achieve the following goals:
\begin{enumerate}
  \setlength{\itemsep}{1pt}
  \setlength{\parskip}{1pt}
  \setlength{\parsep}{0pt}
 \item To gain deeper understanding of multimedia storage and management.
 \item Gain skills in debugging and programming.
 \item Improve teamwork
\end{enumerate}


\section{Environment}


\section{Tools \& Libraries}


\section{Method}

\subsection{Distributed video processing}
How did we extract frames and audio with Hadoop?


\subsection{Image feature extraction}
\subsection{Audio feature extraction}

\subsection{Image Videos Searcher}
\subsection{Audio Videos Searcher}

\subsection{Web Interface}


\section{Results}


\section{Conclusion}


\section{Responsibility table}


\let\thefootnote\relax\footnote{Lin Haiyuang, Wei Chaohou, Dongzhao}

\begin{thebibliography}{99}
  \bibitem{lft}
http://pwhois.org/lft/index.who
  \bibitem{Scamper}
http://www.caida.org/tools/measurement/scamper/
  \bibitem{Gephi}
https://gephi.org/
\end{thebibliography} 

\end{document}          
